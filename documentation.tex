\documentclass[12pt,a4paper]{article}
\usepackage[utf8]{inputenc}
\usepackage[T1]{fontenc}
\usepackage{geometry}
\usepackage{hyperref}
\usepackage{setspace}
\geometry{margin=2.5cm}
\onehalfspacing

\title{Technical Documentation\\
Extractive Question Answering System Based on BERT}
\author{}
\date{}

\begin{document}
\maketitle
\tableofcontents
\newpage

\section{Introduction}

This document describes the technologies used in an extractive \textit{Question Answering} (QA) project, as well as the general structure of the project and how it can be executed. The goal of the project is to develop a system that can answer questions by extracting a relevant fragment from a given text, using modern natural language processing techniques.

The main technologies used are:
\begin{itemize}
    \item BERT and the Hugging Face Transformers library
    \item spaCy
    \item NLTK
    \item scikit-learn
    \item Hugging Face Datasets and Evaluate
\end{itemize}

\section{BERT (Bidirectional Encoder Representations from Transformers)}

BERT is a language model based on the Transformer architecture, developed by Google. Its main characteristics are:

\begin{itemize}
    \item It is a bidirectional model, meaning it analyzes context from both the left and the right of a word simultaneously.
    \item It is pre-trained on very large amounts of text using tasks such as:
    \begin{itemize}
        \item Masked Language Modeling (MLM)
        \item Next Sentence Prediction (NSP)
    \end{itemize}
    \item It can be fine-tuned for specific tasks such as:
    \begin{itemize}
        \item Text classification
        \item Named Entity Recognition
        \item Question Answering
    \end{itemize}
\end{itemize}

In this project, BERT is used for the extractive \textit{Question Answering} task, where the model predicts the start and end positions of the answer within a text.

\section{Hugging Face Transformers}

The Hugging Face Transformers library provides:
\begin{itemize}
    \item Standardized implementations for Transformer models (BERT, RoBERTa, DistilBERT, GPT, etc.)
    \item Optimized tokenizers
    \item Classes for training and evaluation
    \item Ready-to-use pipelines for inference
\end{itemize}

Advantages:
\begin{itemize}
    \item Allows fast loading of pre-trained models
    \item Simplifies the fine-tuning process
    \item Offers a large ecosystem of models and datasets
\end{itemize}

In the project, Hugging Face is used for:
\begin{itemize}
    \item Loading the BERT model
    \item Tokenizing text
    \item Training and evaluation
    \item Running inference through the QA pipeline
\end{itemize}

\section{spaCy}

spaCy is a natural language processing library focused on real-world applications.

Main features:
\begin{itemize}
    \item Tokenization
    \item Sentence segmentation
    \item Lemmatization
    \item Syntactic parsing
    \item Named Entity Recognition (NER)
\end{itemize}

Characteristics:
\begin{itemize}
    \item Very fast and optimized
    \item Provides pre-trained models for multiple languages
    \item Easy to integrate into applications
\end{itemize}

In the project, spaCy is used for:
\begin{itemize}
    \item Cleaning and normalizing text
    \item Removing invalid characters
    \item Light processing without altering text positions (important for extractive QA)
\end{itemize}

\section{NLTK (Natural Language Toolkit)}

NLTK is one of the oldest and most well-known NLP libraries in Python.

Features:
\begin{itemize}
    \item Tokenization
    \item Sentence segmentation
    \item Stemming and lemmatization
    \item Stopwords
    \item Linguistic resources
\end{itemize}

Advantages:
\begin{itemize}
    \item Very good for educational purposes
    \item Offers many classical NLP algorithms
\end{itemize}

In the project, NLTK is used for:
\begin{itemize}
    \item Segmenting context into sentences
    \item Analysis and debugging
    \item Possible splitting of long texts into chunks
\end{itemize}

\section{scikit-learn}

scikit-learn is a classical machine learning library for Python.

It provides:
\begin{itemize}
    \item Classification, regression, and clustering algorithms
    \item Evaluation tools
    \item Data preprocessing utilities
\end{itemize}

In the project, scikit-learn is used optionally for:
\begin{itemize}
    \item Computing additional metrics
    \item Statistical analysis of results
    \item Performance reports
\end{itemize}

\section{Hugging Face Datasets}

Hugging Face Datasets is a library for efficient handling of large datasets.

Features:
\begin{itemize}
    \item Fast loading of standard datasets (SQuAD, GLUE, etc.)
    \item Support for map, filter, shuffle operations
    \item Integration with Transformers
\end{itemize}

In the project it is used for:
\begin{itemize}
    \item Loading the SQuAD dataset
    \item Preprocessing and tokenizing data
    \item Splitting into training and validation sets
\end{itemize}

\section{Hugging Face Evaluate}

Evaluate is a library for computing standard metrics.

Features:
\begin{itemize}
    \item Implementations for NLP metrics
    \item Support for SQuAD, BLEU, ROUGE, etc.
\end{itemize}

In the project it is used for:
\begin{itemize}
    \item Computing Exact Match and F1 metrics for QA
\end{itemize}

\section{Project Structure}

The project follows a standard pipeline:

\begin{enumerate}
    \item Data Loading
    \item Preprocessing
    \item Model Implementation
    \item Evaluation
\end{enumerate}

The logical structure is:

\begin{itemize}
    \item Loading data (e.g., SQuAD or local data)
    \item Cleaning text with spaCy and NLTK
    \item Tokenization with Hugging Face Tokenizer
    \item Training the BERT model
    \item Evaluation using standard metrics
    \item Inference on new data
\end{itemize}

The main file contains:
\begin{itemize}
    \item Preprocessing functions
    \item Model configuration
    \item Training and evaluation
    \item Inference for the user
\end{itemize}

\section{How to Run}

There are two main usage modes:

\subsection{Training and Evaluation}

The project is run in training mode to adapt the model to the SQuAD dataset:

\begin{itemize}
    \item Data is loaded
    \item Preprocessing is applied
    \item The model is trained
    \item Metrics are computed
    \item The resulting model is saved
\end{itemize}

\subsection{Inference}

A pre-trained model is used to answer questions:

\begin{itemize}
    \item The user provides a text and a question
    \item The text is cleaned
    \item The model predicts the answer position
    \item The extracted fragment is displayed
\end{itemize}

\section{Conclusion}

The project demonstrates the integrated use of the most popular modern NLP technologies:
\begin{itemize}
    \item Transformer models (BERT)
    \item The Hugging Face ecosystem
    \item Classical NLP libraries (spaCy, NLTK)
    \item Evaluation tools (Evaluate, scikit-learn)
\end{itemize}

This combination enables rapid development of a robust Question Answering system that is easy to extend and adapt for other natural language processing tasks.

\end{document}
